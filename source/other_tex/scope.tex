\section*{Scope and falsifiability}
This paper develops and tests the core postulates of the \emph{Planck-Cell S/t framework}—%
establishing $\Delta S = \Delta\tau$ and the one-hop-per-tick rule as the minimal causal basis
from which metric relations emerge.  It is not a full textbook; external references are cited
where conventional background is required.  All claims apply to \emph{interior regions} of a
bounded-degree, approximately isotropic adjacency network in which no entity advances more
than one hop per global tick.  The framework is immediately falsified by any of the following:
\begin{enumerate}[label=(\roman*)]
  \item internal-cone violations of the one-hop-per-tick constraint;
  \item interior metric distortion exceeding $f(N)$ under the stated sampling assumptions;
  \item weak-field slope $\hat{\beta}$ outside the preregistered band $[0.97,1.03]$ or consistent residual bias in redshift/clock data.
\end{enumerate}
