\section*{Notation}

\noindent\textbf{Conventions.}  
Time and entropy advance in discrete, \emph{dimensionless ticks}.   
A single primitive cell unit is created per tick, so that
\[
  \Delta S = \Delta \tau.
\]
Physical mapping is used only when required:
\[
  S_{\mathrm{phys}} = k_B\,S, \qquad \tau_s = t_P\,\tau.
\]
Fundamental constants used throughout are the speed of light \(c\), reduced Planck constant \(\hbar\), and Boltzmann constant \(k_B\).

\smallskip
\noindent\textit{Native--SI correspondence (Def.~\ref{def:stitch}).}  
One hop per tick implies \(\mathrm{d}a/\mathrm{d}\tau = 1\).  
Choosing a length per hop \(\ell_0\) and a time per tick \(t_0\) gives
\[
  c = \frac{\ell_0}{t_0}.
\]

\bigskip
\noindent\textbf{Core quantities.}
\begin{description}[leftmargin=2.4em,labelsep=0.8em]
  \item[\(k\)] Tick index (integer). The scale factor \(a(k)\) is defined with respect to ticks; baseline behaviour \(a(k) \propto k^{1/3}\).
  \item[\(\Delta S,\, \Delta\tau\)] Primitive-cell and proper-time tick \emph{counts} (dimensionless), with \(\Delta S = \Delta\tau\).
\end{description}

\noindent\textbf{Graph and geometric parameters (for interior-recovery statements).}
\begin{description}[leftmargin=2.4em,labelsep=0.8em]
  \item[\(d_G(x,y)\)] Graph (hop) distance between vertices \(x\) and \(y\).
  \item[\(\ell_N\)] Typical interior hop length for a sample of size \(N\).
  \item[\(\varepsilon_N,\,\delta_N\)] Mesh non-uniformity and directional-bias parameters (\(\to 0\) as \(N \to \infty\)).
  \item[\(\eta_N\)] Combined small parameter: \(\eta_N := C_1\varepsilon_N + C_2\delta_N.\)
  \item[\(\mathrm{distortion}(N)\)] Bilipschitz distortion of the graph metric relative to the ambient metric:
  \[
    \mathrm{distortion}(N)
      = \sup_{x,y} 
        \max\!\left\{
          \frac{d_G(x,y)}{\|x-y\|},\,
          \frac{\|x-y\|}{d_G(x,y)}
        \right\} - 1.
  \]
\end{description}

\noindent\textbf{Temporal Relativity–specific symbols.}
\begin{description}[leftmargin=2.4em,labelsep=0.8em]
  \item[\(u^\mu,\,a^\mu\)] Four-velocity and four-acceleration along a worldline (ticks parameterize proper time).
  \item[\(\mathcal{C}\)] Causal cone induced by the one-hop-per-tick exchange rule.
  \item[\(\beta\)] Slope in the weak-field clock/redshift relation 
        \(\Delta\nu/\nu = \beta\,\Delta\Phi / c^2.\)
  \item[\(\hat{\beta}\)] Empirical estimator:
        \(\hat{\beta} = (\Delta\nu/\nu)\big/(\Delta\Phi / c^2).\)
\end{description}
