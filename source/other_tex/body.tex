\paragraph{Mini road map.} \emph{(1)} Postulate $\Delta S=\Delta\tau$ and one--hop--per--tick; \emph{(2)} show interior hop geodesics recover distances with a finite--sample rate; \emph{(3)} test weak--field clocks via a preregistered slope $\hat\beta$ with an acceptance band.

% ===========================================================
% Body content begins here. (No packages or random equations.)
% ===========================================================

\section*{Background (i): adjacency \texorpdfstring{$\Rightarrow$}{⇒} metric recovery}
\paragraph{Assumptions.}
(a) finite-speed exchange — updates propagate at most one hop per tick, so causal influence is confined to a finite cone;
(b) interior/bulk regime — vertices considered lie at positive graph distance from the boundary;
(c) non-pathological sampling — degrees remain bounded and adjacency is approximately isotropic (no large directional bias).


\begin{definition}[Causal stitch rate]\label{def:stitch}
Let $a$ denote hop (adjacency) distance and let $\tau$ denote the global tick count (proper-time ticks).
The \emph{causal stitch rate} is the native slope
\begin{equation}
\frac{\mathrm{d}a}{\mathrm{d}\tau}=1,
\end{equation}
which states that influence advances by one hop per tick inside the causal cone.
\end{definition}

\paragraph{Interpretation and SI calibration.}
Choosing physical rulers fixes a length per hop $\ell_0$ and a time per tick $t_0$; then
\begin{equation}
\frac{\mathrm{d}x}{\mathrm{d}t} \;=\; \frac{\ell_0\,\mathrm{d}a}{t_0\,\mathrm{d}\tau} \;=\; \frac{\ell_0}{t_0} \;\equiv\; c.
\end{equation}





Thus \(c\) is the rate the causal frontier stitches per tick, expressed in meters per second.




For later bounds we use the shorthand
\[
\text{distortion}(N)=\sup_{x,y}\max\!\left\{\frac{d_G(x,y)}{\lVert x-y\rVert},\ \frac{\lVert x-y\rVert}{d_G(x,y)}\right\}-1,
\]
i.e., the bilipschitz constant of the adjacency-graph metric relative to the ambient metric, minus one. Informally, it measures the relative stitching/shrinking between hop distances and true distances; $\text{distortion}(N)\to 0$ means recovery is accurate.

\paragraph{Lemma 1 (Interior, explicit distortion).}
Let $\ell_N$ denote the typical interior hop length and let $\eta_N:=C_1\varepsilon_N+C_2\delta_N$, where $\varepsilon_N\to 0$ quantifies mesh non-uniformity and $\delta_N\to 0$ directional bias. Then for interior points $x,y$,
\begin{equation}
(1-\eta_N)\,\lVert x-y\rVert \ \le\ \ell_N\, d_G(x,y) \ \le\ (1+\eta_N)\,\lVert x-y\rVert.
\label{eq:lemma1}
\end{equation}

\paragraph{Theorem 2 (Interior recovery with finite-sample rate).}
Under the assumptions above, the graph metric induced by adjacency graphs recovers the ambient metric in the interior with distortion bounded by
\begin{equation}
\text{distortion}(N)\ \le\ f(N)\ :=\ \frac{C}{\ell_N+\eta_N},
\label{eq:theorem2}
\end{equation}
with $\ell_N\asymp N^{-1/3}$ in 3D and $\eta_N\to 0$ (cf.\ random geometric graph scaling and interior geodesic approximation \cite{penrose2003,tenenbaum2000}). In particular $f(N)\to 0$ as $N\to\infty$.

\paragraph{Remark (related ideas).}
As in the standard adjacency$\to$geometry literature, shortest-path (hop) distances approximate ambient/geodesic distances under interior sampling \cite{tenenbaum2000}. Finite-speed updates induce causal cones and near-isotropy controls bilipschitz distortion; the rate follows from the typical interior hop length $\ell_N\asymp N^{-1/3}$ in 3D together with decay of nonuniformity and directional bias \cite{penrose2003}. \emph{Failure mode:} sufficiently anisotropic or degenerate sampling outside assumptions violates Lemma~\ref{eq:lemma1}.

% ----------------------------
% Weak-field section (baked-in, literature-only)
% ----------------------------
\section*{Result (ii): weak-field clock/redshift check}
\noindent\emph{Units anchor.} In what follows, $c$ enters only through the unit map of Def.~\ref{def:stitch}.
We adopt the standard weak-field prediction that the fractional frequency (or clock-rate) shift is linear in the
Newtonian potential difference, i.e.
\begin{equation}
\frac{\Delta \nu}{\nu} \;=\; \beta\,\frac{\Delta\Phi}{c^{2}}
\label{eq:redshift_model}
\end{equation}

\paragraph{Estimator and preregistered band.}
We estimate the slope by
$\hat\beta := (\Delta\nu/\nu)/(\Delta\Phi/c^2)$ and adopt the acceptance band $\hat\beta\in[0.97,1.03]$;
experiments with reported uncertainties are compared against this band without project-specific fitting.



\paragraph{Canonical checks (numbers from the literature).}
\begin{itemize}
  \item \textbf{Pound--Rebka (1960):} M\"ossbauer $\gamma$ rays up/down $\sim22.6$ m; net fractional
        shift $\Delta\nu/\nu \approx -\,2.56\times10^{-15}\pm0.25\times10^{-15}$, consistent with the predicted value to $\sim10\%$;
        refinements reached $\sim1\%$ by 1964 \cite{poundrebka1960}.
  \item \textbf{Gravity Probe A / Vessot--Levine (1976; 1980 report):} hydrogen maser to $\sim10^4$ km; redshift confirmed
        to $2\times10^{-4}$ relative accuracy \cite{vessot1980}.
  \item \textbf{Hafele--Keating (1972):} circumnavigating aircraft clocks vs.\ ground reference; predicted/observed shifts
        agree within quoted uncertainties (separate GR/SR contributions reported) \cite{hafelekeating1972}.
\end{itemize}

\begin{table}[t]
\centering
\footnotesize
\setlength{\tabcolsep}{4pt}        % tighter columns
\renewcommand{\arraystretch}{1.1}  % modest row spacing
\caption{Weak-field benchmarks (from literature sources).}
\label{tab:weakfield-benchmarks}
\begin{tabular}{@{}p{0.26\linewidth}p{0.26\linewidth}p{0.23\linewidth}p{0.23\linewidth}@{}}
\hline
\textbf{Experiment} &
\textbf{Reported effect} &
\textbf{Predicted} &
\textbf{Measured} \\
\hline
\parbox[t]{\linewidth}{\raggedright Pound--Rebka (1960)} &
\parbox[t]{\linewidth}{\raggedright $\Delta\nu/\nu$ (tower)} &
\parbox[t]{\linewidth}{\raggedright $\sim -2.5\times10^{-15}$} &
\parbox[t]{\linewidth}{\raggedright $-(2.56\pm0.25)\times10^{-15}$ \cite{poundrebka1960}} \\
%
\parbox[t]{\linewidth}{\raggedright Gravity Probe A (1980)} &
\parbox[t]{\linewidth}{\raggedright Fractional redshift test} &
\parbox[t]{\linewidth}{\raggedright $\beta=1$} &
\parbox[t]{\linewidth}{\raggedright $\beta=1\pm2\times10^{-4}$ \cite{vessot1980}} \\
%
\parbox[t]{\linewidth}{\raggedright Hafele--Keating (1972, E)} &
\parbox[t]{\linewidth}{\raggedright Total time shift (ns)} &
\parbox[t]{\linewidth}{\raggedright $-40\pm23$} &
\parbox[t]{\linewidth}{\raggedright $-59\pm10$ \cite{hafelekeating1972}} \\
%
\parbox[t]{\linewidth}{\raggedright Hafele--Keating (1972, W)} &
\parbox[t]{\linewidth}{\raggedright Total time shift (ns)} &
\parbox[t]{\linewidth}{\raggedright $+275\pm21$} &
\parbox[t]{\linewidth}{\raggedright $+273\pm7$ \cite{hafelekeating1972}} \\
\hline
\end{tabular}
\end{table}
\clearpage


% ----------------------------
% Claims/tests (no external script/fig refs)
% ----------------------------
\section*{Claims and tests}
\begin{itemize}
  \item \textbf{A1 (EtLaw) — L0.} Equation: $\Delta S=\Delta\tau$. \emph{Pass:} used as the operational clock.
  \item \textbf{Metric recovery — L3.} Equation: Theorem 2 (rate f(N)). \emph{Pass:} finite-sample bound $f(N)$ under stated assumptions \cite{penrose2003}.
  \item \textbf{Weak-field clocks — L3.} Equation: (6) (weak-field slope). \emph{Pass:} Table~\ref{tab:weakfield-benchmarks} and canonical benchmarks support $\hat\beta\in[0.97,1.03]$ \cite{poundrebka1960,vessot1980,hafelekeating1972}.
\end{itemize}

% ----------------------------
% Reproducibility checklist (document-internal only)
% ----------------------------
\section*{Reproducibility checklist}
\begin{itemize}
  \item \textbf{Self-containment.} All claims in this version are supported by in-document equations, figures, and literature values.
  \item \textbf{Acceptance.} Weak-field slope within the preregistered band; residual patterns qualitatively consistent with cited experiments \cite{poundrebka1960,vessot1980,hafelekeating1972}.
\end{itemize}

% ----------------------------
% Falsifiers (pre-registered)
% ----------------------------
\section*{Falsifiers (pre-registered)}
\begin{itemize}
  \item Any internal cone violations (one-hop-per-tick constraint).
  \item Interior metric distortion exceeding $f(N)$ under the stated sampling assumptions \cite{penrose2003}.
  \item Weak-field slope $\hat\beta$ outside $[0.97,1.03]$ or systematic residual patterns \cite{vessot1980,hafelekeating1972}.
\end{itemize}

% ----------------------------
% Related work (brief)
% ----------------------------
\clearpage
\section*{Related work (brief)}
Thermal time proposes a state-dependent flow of time \cite{connesrovelli1994}; entropic gravity posits gravity as emergent from entropic forces \cite{verlinde2011}; causal sets take order as fundamental \cite{bombelli1987}; graph-geodesic methods recover geometry from adjacency \cite{tenenbaum2000,penrose2003}. Here the novelty is operational: a minimal update ledger $S$ that (i) recovers interior metric structure with explicit finite-sample rates under finite-speed exchange, and (ii) passes a weak-field clock slope test with preregistered bands \cite{poundrebka1960,vessot1980}.

%%% PATCH BEGIN: GR-PL-001
\section*{Clarifications}
\paragraph{What does \(\chi=S/t\) measure?}
\(\chi\) has units s\(^{-1}\): the local tick rate per coordinate time. Gradients bias adjacency transitions.
\paragraph{Flux to Poisson (sketch)}
Stationary divergence of tick-flux equals source density of stiffness, giving \(\nabla^2\chi\propto \rho\). With \(\Phi=\beta\chi\), this is \(\nabla^2\Phi=4\pi G\rho\).
\paragraph{Normalization to \(G\)}
\(\beta C\) fixes SI normalization; one benchmark mass/field sets \(G\) in this weak-field limit.
\paragraph{Equivalence principle}
Inertia and gravitation both scale with the same stiffness parameter, so all bodies share the same drift.
%%% PATCH END: GR-PL-001

\paragraph{Bridge to gravity (teaser).}
Defining the local rate variable $\chi:=S/t$, conservation of $\chi$-flux across adjacency boundaries leads to a Poisson form
$\nabla^2\Phi=4\pi G\rho$ for a potential $\Phi=\beta\chi$, recovering the Newtonian drift field $g=-\nabla\Phi$;
the derivation and assumptions are developed in the companion gravity note.

\begin{tcolorbox}[title=What to measure next]
\textbf{Clocks:} compile $\hat\beta$ across Pound--Rebka, GP-A, HK with a common estimator;\\
\textbf{Vacuum residuals:} search for dispersion/anisotropy/birefringence scaling as $r(\lambda)\propto(\xi_g/\lambda)^p$;\\
\textbf{Phase vs ToF:} interferometer phase compared to time-of-flight over long baselines.
\end{tcolorbox}

\paragraph{Look ahead.}
From these kinematic postulates we next introduce a local stiffness $\omega_0$ (mass as temporal stiffness),
and interpret gravity as the shared drift of the update pattern via $\chi$-flux, before translating thermodynamics and
electromagnetism into the same rate language.